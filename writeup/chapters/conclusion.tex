\chapter{Conclusion} \label{chp:conclusion}
\subsection{Concluding Remarks} \label{sec:conclusion/remarks}

\subsection{Unsolved Problems} \label{sec:conclusion/unsolved}

\subsection{Future Work} \label{sec:conclusion/future-work}
The work that has been presented in this dissertation is a step forwards in dynamic parallelism detection. The framework correctly identifies data-parallel loops, and we have investigated whether the use of bloom filters affects the detection rate. However, there are still significant areas of future work that are possible.

The framework presented is only currently capable of detecting parallelism at the level of loops (\ie, loop-level parallelism). 

Additionally, although the framework \emph{does} correctly detect parallelism at run-time, this information is current under-exploited by the runtime. It is possible, at least theoretically, that the framework could be combined with fellow student Ranjeet Singh's Java-to-OpenCL compiler in a JIT setting to produce a runtime system that dynamically detects parallel loops, recompiles then and executes using OpenCL (either on a CPU or GPU). However, there is an additional downside to this: the advantage (\ie, performance increase) of dynamic recompilation, moving execution to the CPU and then gathering the result must be greater than that of not doing so.

Although thread creation on CPUs is expensive in Java \citep{JSR133}, thread creation on GPUs is low \citep{Mueller2009} (indeed, GPUs need 1000s of threads to operate efficiently in CUDA \citep{Nvidia2011}). In order to properly perform recompilation (or not), a cost model would need to be developed.